\documentclass{article}

\usepackage{amsmath}
\usepackage{amsfonts}
\usepackage{graphicx}
\usepackage{color}

\title{
  Determining the Time at which a Linear Tool Sweep Intersects an Arbitrary
  Point in 3D
}
\date{November 16, 2015}
\author{
  Joseph Coffland\\
  Cauldron Development LLC\\
  joseph@cauldrondevelopment.com
}

\newcommand{\norm}[1]{\lvert #1 \rvert}

\begin{document}
\maketitle

\section{Introduction}
Let vectors $\vec{A}$ and $\vec{B}$ be points in $\mathbb{R}_3$ which define
the linear path $\vec{AB}$ of a cutting tool.  Let $T(r_t,r_b,l)$ represent a
conical tool with top and bottom radi $r_t$ and $r_b$ and length $l$, each in
$\mathbb{R}$.  Let $S$ be the sweep of $T$ as the lowest point of it's center
axis moves from $\vec{A}$ to $\vec{B}$ with the axis of $T$ parallel to the
Z-axis.  Let $\vec{P}$ be an arbitrary point in $\mathbb{R}_3$ near $S$. Let
$A_t$ and $B_t$ be the respective start and end times of $S$.  We wish to
determine if $S$ cuts $\vec{P}$ and if so find $t$, the time at which $S$
firsts cuts $\vec{P}$.

\begin{figure}[!h]
  \centering
  \def\svgwidth{0.5\textwidth}
  \input{conical_tool_point_cut_fig1.eps_tex}
\end{figure}

\section{Find the Point $\vec{Q}$}
First we seek the point $\vec{Q}$ on $\vec{AB}$ where $T$, when centered over
$\vec{Q}$, first cuts $\vec{P}$.  Since $\vec{Q}$ is a point on $\vec{AB}$ we
only need to find the magnitude of $\vec{AQ}$ to fully determine $\vec{Q}$.

Let $\alpha$ be the the magnitude of $\vec{AQ}$.  Then:

\begin{equation*}
\vec{Q} = \alpha \frac{\vec{AB}}{\norm{\vec{AB}}} + \vec{A}
\end{equation*}

Let $\beta = \frac{\alpha}{\norm{\vec{AB}}}$ then we have:

\begin{equation}
\vec{Q} =
[\beta (B_x - A_x) + A_x, \beta (B_y - A_y) + A_y, \beta (B_z - A_z) + A_z]
\end{equation}

There are two ways in which $\vec{P}$ may be cut by $S$.  The first is by
contact with the side of the tool and the second by contact with the upper
or lower tool face.

If $\vec{P}$ is first cut by the side of the tool, then given $\vec{E}$ the
projection of $\vec{P}$ onto the tool axis, the distance between $\vec{P}$ and
$\vec{E}$ is equal to the distance between $\vec{E}$ and $\vec{Q}$ times $T_m$
plus $r_b$.  Where $T_m=\frac{(r_t - r_b)}{l}$ is the slope of $T$.

\begin{equation*}
\norm{\vec{PE}} = \norm{\vec{EQ}} T_m + r_b
\end{equation*}

Note that $\vec{E}$ is just:

\begin{equation*}
\vec{E} = [Qx, Qy, Pz]
\end{equation*}

And $Q_z$ is always less than or equal to $P_z$.  So we have:

\begin{equation}\label{eq2}
\sqrt{(P_x - Q_x)^2 + (P_y - Q_y)^2} = (P_z - Q_z) T_m + r_b
\end{equation}

Substituting equation 1 we get:

\begin{equation*}
\sqrt{(P_x - (\beta (B_x-A_x) + A_x))^2 + (P_y - (\beta (B_y-A_y) + A_y))^2} =
(P_z - (\beta (B_z - A_z) + A_z)) T_m + r_b
\end{equation*}

Solving for $\beta$ we get a big ugly equation.

\section{Spherical Tool}
Consider a spherical tool with radius $r$.  Let $\vec{C}$ be the center of the
sphere with the tool running along $\vec{AB}$ such that $C_z = Q_z + r$ where
$\vec{Q}$ is a point on $\vec{AB}$.  $\vec{C}$ is then just:

\begin{equation*}
\vec{C} = [Q_x, Q_y, Q_z+r]
\end{equation*}

If $S$ cuts $\vec{P}$ then it must first do so when $\vec{P}$ is exactly $r$
away from $\vec{C}$.  Therefore:

\begin{equation*}
\norm{\vec{PC}} = r
\end{equation*}

Substituting in the components of $\vec{C}$ we get:

\begin{equation}
\sqrt{(Px-Qx)^2 + (Py-Qy)^2 + (Pz-(Qz+r))^2} = r
\end{equation}

Substituting equation 1 we get:

\begin{equation}
\sqrt{
  (Px-(\beta (B_x-A_x)+A_x))^2 +
  (Py-(\beta (B_y-A_y)+A_y))^2 +
  (Pz-(\beta (B_z-A_z)+A_z+r))^2} = r
\end{equation}

\section{Time of Cut}
Now the time $t$ when $T$ first cuts $P$ is:

\begin{equation}
t = (B_t - A_t) \beta + A_t
\end{equation}
\end{document}
